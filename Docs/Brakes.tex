\documentclass[../SimBALink.tex]{subfiles}
\begin{document}

\subsection{Inputs and outputs}
	\subsubsection{Inputs}
	\begin{tabular}{ l | l | l  }
		Input					&	Symbol		&	Unit		\\	\hline
		Brake Command			& 	$beta$ 		&	\% \\
		Wheel Speed				&	$\omega_t$	&	rad/s
	\end{tabular}
	
	\subsubsection{Outputs}
	\begin{tabular}{ l | l | l  }
		Output					&	Symbol		&	Unit		\\	\hline
		Brake Force on Tire		&	$F_b$		&	N
	\end{tabular}

\subsection{Background, rationale, modeling strategy}
The brake is modeled as a friction force and a constant that converts $\beta$ to a force.
		\begin{gather}
			F_b = \mu_b \omega_t \beta k_b
		\end{gather}


\subsection{Variables}
	\begin{tabular}{ l | l | l  }
		Var									&	Symbol		&	Unit		\\	\hline
		Brake Coefficient of Friction		&	$\mu_b$		&	 $\frac{N}{rad/s}$ \\
		Force Constant						&	$k_b$		&	 $\frac{N}{\%}$
	\end{tabular}

\subsection{Assumptions}
\begin{itemize}
  \item Brake percentage to friction force is linear
  \item The tire never locks
\end{itemize}

\end{document}