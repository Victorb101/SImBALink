\documentclass[]{article}
\usepackage{gensymb}

\begin{document}

\title{Title}
\author{Author}
\date{Today}
\maketitle

\section{Block diagram}
soon

\section{Inputs and outputs}
	\subsection{Inputs}
	\begin{tabular}{ l | l | l  }
		Input					&	Symbol		&	Unit		\\	\hline
		Reactance current		&	$I_q$		&	A		\\
		Reluctance current		&	$I_d$		&	A		\\
		Coolant inlet temperature	&	$T_{in}$		&	\degree C	\\
		Coolant mass flow rate	&	$\dot{m}_{in}$	&	$\frac{kg}{sec}$	 
	\end{tabular}
	
	\subsection{Outputs}
	\begin{tabular}{ l | l | l  }
		Output					&	Symbol		&	Unit		\\	\hline
		Motor torque				&	$\tau$		&	N m		\\
		Motor back-emf			&	$E_m$		&	V		\\
		Motor temperature			&	$T_{m}$		&	\degree C	\\
		Coolant outlet temperature	&	$T_{out}$		&	\degree C	
	\end{tabular}
	
\section{Background, rationale, modeling strategy}
	\subsection{Torque model}
		The motor is modeled using an empirical torque mapping, which assumes that the torque produced by the motor is a function of the input currents $I_d$ and $I_q$, as well as the motor efficiency:
	
		\begin{equation}
			\tau = f(I_d, I_q) * \eta
		\end{equation}
	
		The motor torque mapping $f(I_d, I_q)$ is determined empirically, and represented in the model as an interpolated 2D lookup table.
	
		The lookup table approach allows for motor physical behavior, such as magnetic saturation and saliency, to be represented without calculating them directly.
	\subsection{Efficiency model}
		The motor efficiency is defined as the ratio between the output mechanical power and the input electrical power:
		
		\begin{equation}
			\eta = \frac{P_m}{P_e} = \frac{V_{app} I_{rms} ??}{\tau \omega}
		\end{equation}
		
		where $V_{app}$ is the applied voltage (line-to-line or phase?), $I_{rms}$ is the motor current (same) and $\omega$ is the motor speed.
		
		The motor efficiency is determined experimentally and represented as a lookup table:
		
		\begin{equation}
			\eta = g(I_{rms}, \omega ?, ...? )
		\end{equation}

\section{Parameters}



\section{Assumptions}

\end{document}